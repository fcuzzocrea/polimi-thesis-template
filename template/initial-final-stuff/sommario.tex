%Il testo delle tesi redatte in lingua straniera dovrà essere introdotto da un ampio estratto in lingua italiana, che andrà collocato dopo l’abstract. 
La possibilità di effetuale manovre a corto raggio in contesti spaziali al fine di compiere missioni di riparazione, rifornimento, rimozione di satelliti non più operativi o detriti spaziali privi di controllo d'assetto sta assumendo sempre più importanza per la programmazione delle future missioni spaziali. Nelle missioni che prevedono l'avvicinamento ad un satellite non-cooperante (cioè non avente alcun tipo di marcatore o supporto che permetta di conoscere a priori il suo assetto relativo), è altresì molto importante che il satellite che effettua le operazioni di avvicinamento sia anche in grado di calcolare con sufficiente precisione l'assetto e la distanza relativa rispetto al \textit{target}, in modo da potersi avvicinare correttamente ed eventualmente afferrare il satellite \textit{target}. In questo contesto, l'impiego di sensori di tipo \textit{vision-based} per il calcolo di assetto e posizioni relativi risulta particolarmente interessante per via delle ridotte dimensioni e del ridotto assorbimento di potenza che caratterizza i sensori di tipo ottico, come può essere una telecamera, rispetto ad altre soluzioni come per esempio quelle basate su sensori \acrshort{lidar}. Per poter implementare tecniche di controllo e navigazione \textit{vision-based}, la posibilità di renderizzare migliaia di immagini dello scenario operativo risulta dunque una tecnologia chiave. Per esempio, le tecniche di \textit{deep learning} si basano sull'analisi di grandi \textit{data-set} di immagini.
Per quanto riguarda le comuni applicazioni terrestri, la disponibilità di un data-set non rappresenta oggigiorno un problema, data la pletora di immagini disponibili. Lo stesso tuttavia non si può dire per quanto riguarda invece le applicazioni spaziali. Le principali ragioni sono rappresentate dalla ogettiva difficoltà di acquisire immagini sufficientemente realistiche e che possano considerarsi come rappresentative di un \textit{data-set}.
Lo scopo di questa tesi è dunque quello di cercare di superare questa limitazione impiegando tecnologie di \textit{ray tracing} al fine di creare uno strumento in grado di produrre automaticamente migliai di immagini di un determinato satellite, noto il suo modello CAD e l'assetto desiderato. Le immagini così prodotto verranno poi utilizzate al fine di implementare e testare un algoritmo \textit{vision-based} atto a stimare assetto e posizione relativi del satellite \textit{target} rappresentato nelle immagini.
Per il raggiungimento dei suddetti obietttivi, la prima cosa da fare è capire come modellare correttamente la scena da rappresentare, andando ad individuare quali sono i corretti parametri ottici da assegnare alle diverse superfici terrestri (oceani, nuvole e terreno) ed alle superfici del satellite (superfici che possono essere di materiale composito, metallico, rivestite da MLI o pannelli solari) che vogliamo rappresentare. Successivamente, è necessario impostare le corrette condizioni di luminosità (ovvero, simulare correttamente la luce solare). Al fine di avere un immagine quanto più realistica possibile è altresì importante simulare correttamente anche il comportamento della camera, in termini di angolo di apertura, risoluzione e rumore intrinseco del sensore. Una volta prodotte le immagini, esse possono essere analizzate utilizzando la metodologia desiderata. Per quanto riguarda questo lavoro, si è scelto di implementare per un algoritmo innovativo per il calcolo di assetto e posizione relativi di un satellite \textit{target} non cooperante, denominato "algoritmo \acrshort{svd}".
L'algoritmo \acrshort{svd} cerca di stimare assetto e posizione relativi del satellite \textit{target} andando ad effettuare un \textit{match} fra le informazioni geometriche del \textit{target} stesso (sotto forma di modello CAD tridimensionale)  - tipicamente presenti all'interno del satellite che effettua l'avvicinamento - con features bidimensionali estratte dall'immagine (sotto forma di segmenti appartenti alla \textit{silhouette} del satellite \textit{target}) tramite tecniche di \textit{edge detection}.
L'algoritmo \acrshort{svd} apporta alcuni miglioramenti rispetto a quello che è lo stato dell'arte, in particolare presenta: una tecnica di eliminazione dei gradienti deboli per determinare una regione di interesse che definisca dove si trovi il satellite \textit{target} all'interno dell'imagine, la definizione di parametri scalabili sulla distanza per quanto riguarda la procedura di \textit{edge detection} ed un processo di catalogazione delle \textit{features} che le classifica in gruppi geometrici, il quale consente di ottenere una consistente riduzione delle possibilità di matching combinando intelligentemente i suddetti gruppi geomtrici. 
La tesi si conclude quindi proponendo al lettore una comparazione qualitativa fra le immagini sintetiche sviluppate per questo progetto  e le immagini appartenenti al data-set SPEED - il primo \textit{data-set} comprendente immagini sintetiche e reali disponibile pubblicamente - e con una validazione preliminare dei risultati ottenuti applicando l'algoritmo \acrshort{svd} sulle prime.