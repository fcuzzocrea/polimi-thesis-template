%«È di cattivo gusto ringraziare il relatore. Se vi ha aiutato ha fatto solo il suo dovere» Umberto Eco, Come si fa una tesi di laurea
Ebbene si, alla fine cel'ho fatta anche io ad arrivare alla fine di questo lungo percorso universitario, iniziato in un soleggiato Settembre del 2009. Di questo devo molto a mio padre e mio fratello, che mi hanno supportato durante tutto questo tempo. Devo però ammettere che non sempre ho creduto di farcela, e cretedemi, non è solo una frase di circostanza, lo ho davvero creduto, sopratutto nei momenti più bui. Arrivare a questo punto non è stato per niente facile e mi è costato tanto, non solo in termini di salute fisica e mentale, ma anche e sopratutto in termini di rapporti umani. Ho conosciuto tante nuove persone che mi hanno aiutato non poco lungo questo percorso ma tante altre le ho perse a causa del mio essere costantemente preoccupato e spaventato per qualche esame, o, ultimamente, per la tesi. Fortunatamente però, sono riuscito a conoscere tante persone diverse che, chi più chi meno, mi hanno dato qualcosa e mi sono state vicine. Come non ringraziare il mitologico Flavio, che mi è sempre stato vicino, o come non ricordare Aureliano, Federico, Mattia, Umberto e Trevis che mi sono stati vicini sin dal lontano 2012, o anche Viviana e Andrea, sempre presenti per me quando la mia vita sentimentale andava a rotoli. Per non parlare dele famose "pause accademiche" di Stefano e Davide durante le lunghe sessioni di studio in biblioteca !! E' anche doveroso da parte mia dover ringraziare Alfonso e Benedetto, con i quali ho condiviso tutto il percorso della laurea Magistrale. A tutte queste persone conosciute in Università voglio dedicare il mio grazie per avermi sempre aiutato lungo tutto il mio percorso universitario. Un pensiero speciale e la mia immensa gratitudine voglio invece dedicarli a Jacopo Guarneri ed Ilaria Cannizzaro per il supporto tecnico e sopratutto morale che mi hanno fornito durante questa avventura trascorsa insieme in D-Orbit, soprattutto mentre si discuteva di "rotazioni". Ancora qualche riga voglio spenderla per mandare un pensiero ad Amro, Anas, Antonio, Emilio, Mayra e tutti i ragazzi che ho conosciuto in biblioteca durante la stesura finale di questo lavoro. Grazie a tutti voi ogni singola goccia di sudore spesa su questa tesi è stata accompagnata da un sorriso. Un debito ringraziamento voglio rivolgerlo a tutti i professori che in questi anni di Politecnico mi sono stati vicini, in particolare, i proff. Colombo, Quartapelle e Mantegazza meritano i miei ringraziamenti più sentiti per avermi aiutato a capire che tipo di strada intraprendere.
Infine, ma non certo ultimi per ordine di importanza, voglio ringraziare di cuore Cecilia e la sua famiglia per essermi stati vicini, per avermi sopportato ed avermi voluto bene negli anni passati.\\


Francescodario Cuzzocrea, \today, Milano