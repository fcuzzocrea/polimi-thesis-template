%«È di cattivo gusto ringraziare il relatore. Se vi ha aiutato ha fatto solo il suo dovere» Umberto Eco, Come si fa una tesi di laurea
Ebbene si, alla fine cel'ho fatta anche io ad arrivare alla fine di questo lungo percorso universitario, iniziato in un soleggiato Settembre del 2009. Di questo devo molto a mio padre e mio fratello, che mi hanno supportato durante tutto questo tempo. Devo però ammettere che non sempre ho creduto di farcela, e cretedemi, non è solo una frase di circostanza, lo ho davvero creduto, sopratutto nei momenti più bui. Arrivare a questo punto non è stato per niente facile e mi è costato tanto, non solo in termini di salute fisica e mentale, ma anche e sopratutto in termini di rapporti umani. Diversi sono i rapporti che ho perso a causa del mio essere costantemente preoccupato e spaventato per qualche esame, o, ultimamente, per la tesi. Fortunatamente però, sono anche riuscito a circondarmi di tante altre persone che, chi più chi meno, mi hanno dato qualcosa e mi sono state vicine. In particolare, un pensiero speciale e la mia immensa gratitudine vanno a Ilaria Cannizzaro e Jacopo Guarneri per il supporto tecnico e sopratutto morale che mi hanno fornito durante questa avventura trascorsa insieme in D-Orbit, specie mentre si discuteva di \textit{"rotazioni"}. Come non ringraziare inoltre il mitologico Flavio, che mi è sempre stato vicino, sopratutto nei momenti più difficili della vita, e come non ricordare Aureliano, Federico, Mattia, Umberto e Trevis che mi hanno supportato sin dal lontano 2012. Ma anche Viviana e Andrea, sempre presenti con me nella Biblioteca di Aerospaziale a studiare! Per non parlare dele famose "pause accademiche" di Stefano e Davide durante le lunghe sessioni di studio pre-esame!! E' anche doveroso da parte mia dover ringraziare Alfonso e Benedetto, con i quali ho condiviso tutto il percorso della Laurea Magistrale. A tutte queste persone conosciute in Università voglio dedicare il mio grazie per essermi state vicine lungo tutto il mio percorso universitario e per avermi lasciato, chi più chi meno, qualcosa di loro. Ancora qualche riga voglio spenderla per mandare un pensiero speciale ad Amro, Anas, Antonio, Emilio, Gian Marco, Mayra e tutti i ragazzi che ho conosciuto in Biblioteca durante la stesura finale di questo lavoro. Grazie a tutti voi, ogni singola goccia di sudore spesa su questa tesi è stata accompagnata da un sorriso. Un debito ringraziamento voglio inoltre rivolgerlo a tutti i professori che in questi anni di Politecnico mi sono stati vicini, in particolare, i proff. Colombo, Mantegazza e Quartapelle meritano i miei ringraziamenti più sentiti per avermi aiutato a capire che tipo di strada intraprendere quando le mie idee erano confuse.
\newpage
Infine, ma non certo ultimi per ordine di importanza, voglio ringraziare Cecilia e la sua famiglia per essermi stati accanto negli anni passati.

\vspace{\baselineskip}

\textit{Francescodario Cuzzocrea}, \today, Milano